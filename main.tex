\documentclass[11pt, hidelinks]{article}
\usepackage[a4paper,tmargin=1in,bmargin=1in,lmargin=1.25in,rmargin=1.25in]{geometry}
\usepackage[T1]{fontenc}
\usepackage[ngerman]{babel}
\usepackage[dvipsnames]{xcolor}
\usepackage{amsthm}
\usepackage{amsmath, amsfonts, amssymb}
\usepackage{euler}
\usepackage{tikz}
\usepackage{tkz-euclide}
\usepackage[unicode]{hyperref}
\usepackage[all]{hypcap}
\usepackage{fancyhdr}
\usepackage{mathtools}
\usepackage[singlespacing]{setspace}
\usepackage{enumitem}
\setlist[itemize]{noitemsep, nolistsep}
\setlist[enumerate]{noitemsep, nolistsep}

\usepackage[osf,largesc,theoremfont]{newpxtext}
\usepackage[scaled=1.03]{inconsolata}
\usepackage[euler-digits,euler-hat-accent]{eulervm}

\usetikzlibrary{angles,calc, decorations.pathreplacing}

\definecolor{carminered}{rgb}{1.0, 0.0, 0.22}
\definecolor{capri}{rgb}{0.0, 0.75, 1.0}
\definecolor{brightlavender}{rgb}{0.75, 0.58, 0.89}

\newcommand{\h}{\mathcal{H}}
\newcommand{\B}{\mathcal{B}}
\newcommand{\spec}{\operatorname{spec}}
\newcommand{\ind}{\operatorname{ind}}
\newcommand{\im}{\operatorname{im}}
\newcommand{\coker}{\operatorname{coker}}
\newcommand{\on}{{n \geq 1}}

\newtheorem{conj}{Vermutung}
\numberwithin{conj}{section}
\newtheorem{definition}[conj]{Definition}
\newtheorem{remark}[conj]{Anmerkung}
\newtheorem{example}[conj]{Beispiel}
\newtheorem{theorem}[conj]{Satz}
\newtheorem{lemma}[conj]{Lemma}
\newtheorem{proposition}[conj]{Proposition}
\newtheorem{corollary}[conj]{Korollar}
\newtheorem{assume}[conj]{Annahme}

\title{\textbf{Beschränkte Fremdholmoperatoren und deren Fremdholmindex auf separablen Hilberträumen}}
\author{
\textbf{Luciano Melodia} \\
Seminar zur Funktionalanalysis \\
Lehrstuhl für Mathematische Physik und Operatoralgebren \\
Friedrich-Alexander Universität Erlangen-Nürnberg \\
\texttt{luciano.melodia@fau.de}}
\date{\today}
\allowdisplaybreaks
\begin{document}
\hypersetup{bookmarksnumbered=true}
\maketitle

\begin{abstract}
Ein Fredholmoperator \(T\) ist ein Operator zwischen Hilberträumen, für den die Lösungen des inhomogenen linearen Problems \(T \varphi = \phi\) durch „endlich viele Daten“ charakterisiert werden können, ähnlich wie im endlich-dimensionalen Fall. Konkret bedeutet dies, dass der Kern \(\ker(T)\) endlich-dimensional ist, d.h. es existiert eine endliche Basis \(\{\varphi_1, \dots, \varphi_n\}\) für \(\ker(T)\). Ebenso ist der Kokern \(\operatorname{coker}(T) = \mathcal{Y} / \operatorname{Im}(T)\) endlich-dimensional, sodass es endlich viele lineare Funktionale \(\{S_1, \dots, S_k\}\) auf \(\mathcal{Y}\) gibt, welche der Bedingung \(\phi \in \operatorname{Im}(T)\) genügen, welche äquivalent ist zu \(S_1 \phi = \dots = S_k \phi  = 0\). Die Gleichung \(T \varphi = \phi\) besitzt genau dann eine Lösung, wenn \(S_1 \phi = \dots = S_k \phi = 0\). Falls eine Lösung existiert, bildet die Lösungsmenge eine endliche affine Untermenge, gegeben durch \(\varphi_0 + \langle \varphi_1, \dots, \varphi_n \rangle\), wobei \(\varphi_0\) eine spezielle Lösung des inhomogenen Problems ist, d.h. \(T \varphi_0 = \phi\). Der Index von \(T\) ist definiert als \(\operatorname{ind}(T) = \dim(\ker(T)) - \dim(\operatorname{coker}(T)) = n - k\). Ein entscheidender Aspekt des Index ist seine Invarianz gegenüber kompakten Störungen und seine Stetigkeit auf der offenen Menge der Fredholmoperatoren. Die Zusammenhangskomponenten dieser Menge stehen in Bijektion mit dem Index. Im endlich-dimensionalen Fall, wenn \(T: \mathbb{C}^n \to \mathbb{C}^n\) eine lineare Abbildung ist, ist \(T\) automatisch ein Fredholmoperator mit Index Null. Sei \(\lambda \in \mathbb{C}\), dann ist der Kern von \(T - \lambda I\) für fast alle \(\lambda\) trivial und \(T - \lambda I\) surjektiv, außer wenn \(\lambda\) ein Eigenwert von \(T\) ist. In diesem Fall erhöht sich die Dimension des Kerns (um die geometrische Vielfachheit von \(\lambda\)), während die Dimension des Bildes gemäß dem Satz von Rang und Kern (Rang-Nullitätssatz) um denselben Betrag abnimmt, sodass der Index erhalten bleibt. Im unendlich-dimensionalen Fall gilt der Satz von Rang und Kern im Allgemeinen nicht, und es besteht im Normalfall kein einfacher Zusammenhang zwischen den Dimensionen von \(\ker(T)\) und \(\operatorname{Im}(T)\). Für Fredholmoperatoren \(T: \mathcal{H} \to \mathcal{Y}\) mit Index \(L\) jedoch beschreibt \(L\) die Differenz zwischen den Dimensionen des Kerns und des Kokerns. Wenn \(\mathcal{H} = \mathcal{Y}\) ist und \(T - \lambda I\) für alle \(\lambda\) Fredholm bleibt, dann wird der Sprung in der Dimension des Kerns von \(T - \lambda I\), wenn \(\lambda\) ein Eigenwert ist, sowie die entsprechende Abnahme der Dimension des Bildes, durch den Index \(L\) bestimmt. Für \(L = 0\) gleichen sich diese Sprünge aus. Ist \(L > 0\), so kann \(T - \lambda I\) nicht injektiv sein; falls \(T - \lambda I\) surjektiv ist, beträgt die Dimension des Kerns genau \(L\). Generell kann die Dimension des Kerns von \(L\) auf \(L + r\) (mit \(r > 0\)) anwachsen, wobei die Dimension des Bildes um \(r\) abnimmt, um dies zu kompensieren. Dies bedeutet, dass die Dimension des Kokerns ebenfalls um \(r\) zunimmt. Wenn \(L < 0\) ist, kann \(T - \lambda I\) nicht surjektiv sein, und eine analoge Analyse ergibt sich.
\end{abstract}

\begin{Large}
\tableofcontents
\end{Large}

\onehalfspacing

\section{Fredholmoperatoren und Fredholmindex}
Ein Fredholmoperator ist ein beschränkter linearer Operator \( T: \h \to \mathcal{Y} \) zwischen separablen Hilberträumen, der eine fundamentale Rolle in der Theorie der linearen Operatoren spielt. Die wesentliche Eigenschaft eines Fredholmoperators ist, dass er „fast invertierbar“ ist, was bedeutet, dass er sich wie ein Operator verhält, der bis auf endliche Störungen invertierbar ist. Konkret verlangt man von einem Fredholmoperator, dass sowohl der Kern als auch der Kokern endlich-dimensional sind. Der Kern eines Operators \( T \), geschrieben als \( \ker(T) = \{ \varphi \in \mathcal{H} \mid T\varphi = 0 \} \), ist der Raum der Elemente, die auf Null abgebildet werden. Dass der Kern endlich-dimensional ist, bedeutet, dass die Anzahl linear unabhängiger Elemente in \( \ker(T) \) endlich ist. Der Kokern ist definiert als der Quotientenraum \( \mathcal{Y} / \operatorname{Im}(T) \), also der Raum der Elemente, die nicht im Bild des Operators liegen. Wenn dieser Raum endlich-dimensional ist, bedeutet das, dass der Operator „nahe“ an einem surjektiven Operator liegt, dessen Bild den gesamten Raum \( \mathcal{Y} \) abdeckt. Der Index eines Fredholmoperators ist eine sehr wichtige Zahl, die seine algebraischen Eigenschaften beschreibt. Er ist definiert als die Differenz zwischen der Dimension des Kerns und der Dimension des Kokerns:
\[
\operatorname{ind}(T) = \dim(\ker(T)) - \dim(\operatorname{coker}(T)).
\]
Dieser Index ist eine ganze Zahl, und was ihn besonders macht, ist seine Stabilität unter kompakten Störungen. Das bedeutet, dass der Index eines Fredholmoperators unverändert bleibt, wenn man den Operator um einen kompakten Operator ergänzt, was eine zentrale Eigenschaft in vielen Anwendungen der Theorie der Fredholmoperatoren ist, wie etwa in der Theorie der elliptischen Differentialoperatoren. Diese Stabilität ist besonders nützlich in Situationen, in denen man die invertierbaren Eigenschaften eines Operators durch kleine oder endliche Störungen nicht verlieren möchte. Der Index eines Fredholmoperators fasst also die Abweichung von der Invertierbarkeit zusammen. Ein Fredholmoperator mit Index Null ist fast invertierbar, während ein Operator mit nichttrivialem Index eine gewisse Abweichung davon aufweist, die jedoch kontrollierbar ist, da der Kern und der Kokern endlich-dimensional sind.

\begin{definition}
    Ein Operator $T \in \mathcal{B}(\h,\mathcal{Y})$ ist Fredholmsch genau dann, wenn
    \begin{enumerate}
        \item $\dim(\ker(T)) < \infty$,
        \item $\dim(\ker(T^\ast)) < \infty$,
        \item $\im(T)$ ist abgeschlossen in $\mathcal{Y}$.
    \end{enumerate}
\end{definition}

Die Menge der Fredholmoperatoren wird als $\mathcal{FB}(\h,\mathcal{Y})$ geschrieben.

\begin{definition}
    Der Index eines Fredholmoperators $T \in \mathcal{FB}(\h,\mathcal{Y})$ ist
    \begin{align}
        \ind(T) &= \dim(\ker(T)) - \dim(\ker(T^\ast)).
    \end{align}
    Da $\ker(T^\ast) = \im(T)^\perp$ und $\im(T)$ für einen Fredholmoperator abgeschlossen sind, können wir den Index auch umschreiben als
    \begin{align}
        \ind(T) &= \dim(\ker(T)) - \dim(\mathcal{Y}/\im(T)) \nonumber\\
                &= \dim(\ker(T)) - \dim(\coker(T)).
    \end{align}
\end{definition}

\begin{example}\noindent
    \begin{enumerate}
        \item Sei $\mathbf{1}: \h \to \h$ die Identität auf einem endlichdimensionalen Hilbertraum. Dann ist $\mathbf{1}$ Fredholmsch, denn $\dim(\ker(\mathbf{1})) = 0 < \infty$, $\dim(\ker(\mathbf{1}^\ast)) = \dim(\ker(\mathbf{1})) = 0 < \infty$ und $\im(\mathbf{1}) = \h$ ist abgeschlossen. Somit ist der Index $\operatorname{ind}(\mathbf{1}) = \dim(\ker(\mathbf{1})) - \dim(\operatorname{coker}(\mathbf{1})) = 0 - 0 = 0$.
        \item Sei \( T = \mathbf{1} + K \), wobei \( \mathbf{1} \) die Identität und \( K \) ein kompakter Operator ist. \( T \) ist demnach beschränkt und linear, da \( \mathbf{1} \) und \( K \) beschränkte lineare Operatoren sind. Nach Korollar \ref{summeFredholm} ist $T$ Fredholmsch. Es gilt $\operatorname{ind}(T) = \dim(\ker(T)) - \dim(\coker(T)) = 0 - 0 = \operatorname{ind}(\mathbf{1})$.
        \item Sei $S: \ell^2(\mathbb{N}) \to \ell^2(\mathbb{N})$ der rechte Shift-Operator, definiert durch $S(x_n) = x_{n+1}$ für $n \geq 2$ und $S(x_n) = 0$ für $n=1$. Dann ist $\ker(S) = \{0\}$ also $\dim(\ker(S)) < \infty$ und $\im(S) = \{ (x_n)_{n \geq 1} \in \ell^2(\mathbb{N}) \; \vert \; x_1 = 0\}$. Somit sehen wir, dass $\h/\im(S) = \langle (\delta_{1n})_{n \geq 1} \rangle_{\mathbb{C}}$ und damit $\dim(\h/\im(S)) = 1 < \infty$ da $\{(\delta_{mn})_{n \geq 1}\}_{m \geq 1}$ eine Othonormalbasis für $\ell^2(\mathbb{N})$ definiert. Sei $(x_n)^{n \geq 1}_k \to (x_n)_{n \geq 1}$ für $k \to \infty$ eine Folge in $\im(S)$. Dann gilt
        \begin{align}
            &\left\vert \langle (\delta_{1n})_{n \geq 1}, (x_n)_{n \geq 1} \rangle_{\ell^2(\mathbb{N})} - \langle (\delta_{1n})_{n \geq 1}, (x_n)^{n \geq 1}_k \rangle_{\ell^2(\mathbb{N})} \right\vert = \nonumber\\
            & \vert x_1 - (x_1)_k \vert \nonumber \xrightarrow[]{k \to \infty} \vert x_1 - 0 \vert = \vert 0 - 0 \vert = 0,
        \end{align}
        also ist $(x_n)_{n \geq 1} \in \im(S)$ und das Bild von $S$ somit abgeschlossen. Der Index von \( S \) ist also $\operatorname{ind}(S) = \dim(\ker(S)) - \dim(\operatorname{coker}(S)) = 0 - 1 = -1.$
        \item Sei \( H = L^2(\mathbb{R}) \) der Hilbertraum der quadratintegrierbaren Funktionen auf \( \mathbb{R} \), und sei $m(x) = 1 + \frac{1}{1 + x^2}$ eine beschränkte Funktion, da $\|m\|_\infty = \sup_{x \in \mathbb{R}} |m(x)| = 1 + 1 = 2 < \infty$. Da \( m(x) = 1 + \frac{1}{1 + x^2} \geq 1 \) für alle \( x \in \mathbb{R} \), folgt \( m(x)^{-1} \leq 1 \) und da alles strikt positiv ist auch \( \vert m(x)^{-1} \vert \leq 1 \), also ist \( m(x)^{-1} \) beschränkt. Beachte, dass \( m(x) \geq 1 \) für alle \( x \in \mathbb{R} \), also ist \( m(x) \) überall von Null verschieden. Der Multiplikationsoperator \( M_m \) ist definiert durch $(M_m f)(x) = m(x) f(x)$ für $f \in L^2(\mathbb{R})$. Wir möchten zeigen, dass der Operator \( M_m \) ein Fredholmoperator ist. Zunächst betrachten wir den Kern
        \begin{align}
        \ker(M_m) = \left\{ f \in L^2(\mathbb{R}) \mid m(x) f(x) = 0 \text{ fast überall auf } \mathbb{R} \right\}.
        \end{align}
        Da \( m(x) \geq 1 > 0 \) für alle \( x \in \mathbb{R} \), impliziert \( m(x) f(x) = 0 \), dass \( f(x) = 0 \) fast überall auf \( \mathbb{R} \). Somit ist $\ker(M_m) = \{0\}$, der Kern ist also trivial. Das Bild ist
        \begin{align}
            \operatorname{Im}(M_m) = \left\{ g \in L^2(\mathbb{R}) \mid \exists f \in L^2(\mathbb{R}) \text{ mit } g(x) = m(x) f(x) \text{ fast überall} \right\}.
        \end{align}
        Somit ist der Operator \( M_m \) invertierbar mit der Umkehrabbildung $(M_m^{-1} g)(x) = m(x)^{-1} g(x)$, die ebenfalls beschränkt ist. Da \( M_m \) beschränkt und invertierbar ist, ist \( \operatorname{Im}(M_m) = L^2(\mathbb{R}) \) und der Quotientenraum \( L^2(\mathbb{R}) / \operatorname{Im}(M_m) =  \{0\}\). Da sowohl der Kern als auch das Koker endlichdimensional sind (Dimension Null), ist \( M_m \) ein Fredholmoperator mit Index Null, da $\im(T) = L^2(\mathbb{R})$ abgeschlossen ist.
    \end{enumerate}
\end{example}


\section{Spektraltheorie kompakter Operatoren}
Ein Hilbertraum heißt separabel, wenn er eine abzählbare Orthonormalbasis besitzt. Wir betrachten separable komplexe Hilberträume $\h, \mathcal{Y}$ unendlicher Dimension mit assoziierter Norm $\Vert \varphi \Vert_{\h} = \langle \varphi, \varphi \rangle^{\frac{1}{2}}_{\h}$ für $\varphi \in \h$ bzw. $\Vert \varphi \Vert_{\mathcal{Y}} = \langle \varphi, \varphi \rangle^{\frac{1}{2}}_{\mathcal{Y}}$ für $\varphi \in \mathcal{Y}$. Für einen linearen Operator $T: \h \rightarrow \mathcal{Y}$ ist dessen Operatornorm wie folgt definiert:
\begin{align}
    \Vert T \Vert = \sup_{\varphi \neq 0} \frac{\Vert T\varphi \Vert_{\mathcal{Y}}}{\Vert\varphi\Vert_\h} = \sup_{\Vert\varphi\Vert_\h = 1} \Vert T\varphi\Vert_{\mathcal{Y}}.
\end{align}
Der Operator $T$ heißt beschränkt, falls $\Vert T \Vert < \infty$. Die Menge aller beschränkten linearen Hilbertraumoperatoren $\h \to \mathcal{Y}$ wird mit $\B(\h,\mathcal{Y})$ bezeichnet. Für $\h = \mathcal{Y}$ schreibt man auch $\B(\h)$. Für $\B(\h)$ ist $\ast: \h \to \h$, $A \mapsto A^\ast \ldots$
\begin{enumerate}
    \item[(1)] eine Involution, d.h. $(T^\ast)^\ast = T$,
    \item[(2)] antimultiplikativ $(TS)^\ast = T^\ast S^\ast$,
    \item[(3)] antilinear $(\lambda T + \mu S)^\ast = \overline{\lambda} T^\ast + \overline{\mu} S^\ast$,
    \item[(4)] für die Operatornorm gilt $\Vert TS  \Vert \leq \Vert T \Vert \Vert S \Vert$ und
    \item[(5)] $\B(\h)$ ist eine $C^\ast$-algebra, d.h. $\Vert T \Vert^2 = \Vert T^\ast T \Vert$.
\end{enumerate}
Als Hilbertraum ist $\B(\h)$ vollständig und die Adjunktion ist eine Isometrie $\Vert T^\star \Vert = \Vert T \Vert$.

\begin{definition}
    Der Einheitsball in $\h$ ist die Menge $\mathbb{B}_\h := \{ \varphi \in \h \; \vert \; \Vert \varphi \Vert_\h \leq 1 \}$.
\end{definition}

\begin{proposition}
    $\mathbb{B}_\h$ ist kompakt genau dann, wenn $\h$ endliche Dimension hat.
\end{proposition}

\begin{proof}
    Sei $\h$ ein endlichdimensionaler Hilbertraum. Damit ist $\h \cong \mathbb{C}^{\vert B_\h \vert}$ für eine insbesondere endliche Basis $B_\h$. Dann gilt für alle $\varphi \in \mathbb{B}_\h$, dass $\Vert \varphi \Vert_\h \leq 1$ nach Definition des Einheitsballs. Da $\mathbb{B}_\h$ abgeschlossen und beschränkt ist, ist diese Menge auch kompakt, nach Heine-Borel.

    Sei $\h$ unendlichdimensional und separabel. Dann hat $\h$ eine Basis $B_\h$ mit abz. vielen Elementen. Wir betrachten ein abzählbares Orthonormalsystem $B^\sharp_\h = \{b_n\}_\on$. Dann ist $\Vert b_n \Vert_\h = 1$ und $b_n \in \mathbb{B}_\h$ für alle $n \geq 1$. Aufgrund der Orthonormalität gilt $\Vert b_n - b_m \Vert_\h = \sqrt{2}$ für alle $n \neq m$ und $b_n,b_m \in B^\sharp_\h$. Also ist keine Teilfolge von $(b_n)_\on$ Cauchy. Somit ist $(b_n)_\on$ eine Folge ohne konvergente Teilfolge, aber in $\mathbb{B}_\h$ enthalten. Damit ist $\mathbb{B}_\h$ nicht kompakt.
\end{proof}

Wir erinnern an die Klasse der kompakten Operatoren. Sie werden definiert als beschränkte lineare Operatoren, welche $\mathbb{B}_\h$ auf eine präkompakte Menge abbilden. Eine Menge heißt präkompakt, wenn ihr Abschluss kompakt ist.

\begin{definition}
    Ein Operator $K \in \mathcal{B}(\h,\mathcal{Y})$ heißt kompakt genau dann, wenn das Bild des Einheitsballs $K(\mathbb{B}_\h)$ präkompakt ist, also einen kompakten Abschluss besitzt. Die Menge aller kompakten Operatoren von $\h$ nach $\mathcal{Y}$ wird als $\mathcal{K}(\h,\mathcal{Y})$ geschrieben. Für $\h=\mathcal{Y}$ schreiben wir $\mathcal{K}(\h) \coloneq \mathcal{K}(\h,\h)$.
\end{definition}

\begin{remark}
    Es gilt $\mathcal{K}(\h,\mathcal{Y}) \subset \mathcal{B}(\h,\mathcal{Y})$.
\end{remark}

\begin{remark}
Zur kurzen Erinnerung: Jede beschränkte Folge in einem kompakten metrischen Raum besitzt eine konvergente Teilfolge.
\end{remark}

\begin{proposition}
    $K \in \mathcal{B}(\h,\mathcal{Y})$ ist kompakt genau dann, wenn für jede beschränkte Folge $(\varphi_n)_{n \geq 1}$ in $\h$ die Folge $(K\varphi_n)_{n \geq 1}$ in $\mathcal{Y}$ eine konvergente Teilfolge hat.
\end{proposition}

\begin{proof}
Sei $K \in \mathcal{B}(\h,\mathcal{Y})$ ein beschränkter Operator.

Angenommen, $K$ ist kompakt. Das bedeutet, dass $K$ eine beschränkte Menge in $\h$ auf eine präkompakte Menge in $\mathcal{Y}$ abbildet. Sei $(\varphi_n)_{n \geq 1}$ eine beschränkte Folge in $\h$. Da $K$ kompakt ist, ist $\{K\varphi_n \; \vert \; n \geq 1\}$ enthalten in einer präkompakten Menge $K' \subset \mathcal{Y}$. Nach Definition der Präkompaktheit besitzt $(K\varphi_n)_{n \geq 1}$ eine konvergente Teilfolge in $K'$.

Angenommen, für jede beschränkte Folge $(\varphi_n)_{n \geq 1}$ in $\h$ besitzt $(K\varphi_n)_{n \geq 1}$ eine konvergente Teilfolge in $\mathcal{Y}$. Sei $B \subset \h$ eine beschränkte Menge. Dann können wir eine Folge $(\varphi_n)_{n \geq 1} \subset B$ wählen, die eine konvergente Teilfolge $(K\varphi_{n_k})_{k \geq 1}$ in $\mathcal{Y}$ hat. Also ist $(K\varphi_{n_k})_{k \geq 1}$ eine konvergente Teilfolge der Folge $(K\varphi_{n})_{n \geq 1}$. Dies zeigt, dass $K(B)$ präkompakt ist, also ist $K$ kompakt.
\end{proof}

\begin{theorem}
    \label{schauder}
    Für $K \in \mathcal{K}(\h,\mathcal{Y}), A \in \mathcal{B}(\mathcal{Y},\mathcal{Z})$ und $B \in \mathcal{B}(\mathcal{Z},\h)$, wobei $\mathcal{Z}$ ein anderer separabler Hilbertraum ist, sind die Kompositionen $AK \in \mathcal{B}(\h,\mathcal{Z})$ und $KB \in \mathcal{B}(\mathcal{Z},\mathcal{Y})$ kompakt. Weiterhin ist der adjungierte Operator $K^\ast \in \mathcal{B}(\mathcal{Y},\h)$ kompakt.
\end{theorem}

\begin{proof}
Sei \( K \in \mathcal{K}(\h, \mathcal{Y}) \) ein kompakter Operator und \( \h \) und \( \mathcal{Y} \) separable Hilberträume. Sei \( (\varphi_n)_{n \geq 1} \) eine beschränkte Folge in \( \h \). Da \( K \) kompakt ist, besitzt \( (K\varphi_n)_{n \geq 1} \) eine in \( \mathcal{Y} \) konvergente Teilfolge, sagen wir \( (K\varphi_{n_k})_{k \geq 1} \), mit \( K\varphi_{n_k} \xrightarrow[]{k \rightarrow \infty} \phi \in \mathcal{Y} \). Da \( A \in \mathcal{B}(\mathcal{Y}, \mathcal{Z}) \) beschränkt und damit stetig ist, folgt, dass \( A K\varphi_{n_k} \xrightarrow[]{k \rightarrow \infty} A\phi \in \mathcal{Z} \). Somit hat die Folge \( (AK\varphi_n)_{n \geq 1} \) eine konvergente Teilfolge, und damit ist \( AK \) kompakt.

Sei \( (\varphi_n)_{n \geq 1} \) eine beschränkte Folge in \( \mathcal{Z} \). Da \( B \in \mathcal{B}(\mathcal{Z}, \mathcal{Y}) \) beschränkt ist, ist \( (B\varphi_n)_{n \geq 1} \) eine beschränkte Folge in \( \mathcal{Y} \). Da \( K \) kompakt ist, besitzt \( (KB\varphi_n)_{n \geq 1} \) eine konvergente Teilfolge in \( \mathcal{Y} \). Somit ist \( KB \) kompakt.

Betrachte eine beschränkte Folge $(\varphi_n)_{n \geq 1} \subset \mathcal{Y}$, mit $\Vert \varphi_n - \varphi_m \Vert_{\mathcal{Y}} \leq M$ für alle $n,m \in \mathbb{N}$. Da $K^\ast$ beschränkt ist hat $(KK^\ast\varphi_n)_{n \geq 1} \subset \mathcal{Y}$ eine konvergente Teilfolge $(KK^\ast\varphi_{n_k})_{n \geq 1}$. Es gilt nun mittels Skalarprodukt
\begin{align}
    \Vert K^\ast \varphi_{n_k} - K^\ast\varphi_{n_l} \Vert^2_\h &= \langle KK^\ast (\varphi_{n_k} - \varphi_{n_l}), \varphi_{n_k} - \varphi_{n_l} \rangle \\
    &\leq \Vert KK^\ast (\varphi_{n_k} - \varphi_{n_l}) \Vert_{\mathcal{Y}} \cdot \Vert \varphi_{n_k} - \varphi_{n_l} \Vert_{\mathcal{Y}} \\
    &\leq M \Vert KK^\ast (\varphi_{n_k} - \varphi_{n_l}) \Vert_{\mathcal{Y}} \to 0.
\end{align}
Also ist $(K^\ast \varphi_{n_k})_{k \in \mathbb{N}}$ eine Cauchy-Folge in einem Hilbertraum und aufgrund von dessen Vollständigkeit auch konvergent. Also hat $(K^\ast \varphi_{n})_{n \geq 1}$ eine konvergente Teilfolge für jede beschränkte Folge $(\varphi_n)_{n \geq 1} \subset \mathcal{Y}$, weshalb $K^\ast$ kompakt ist.
\end{proof}

\begin{proposition}
Sei \( \mathcal{A} \) eine \( C^\ast \)-Algebra und \( I \) ein zweiseitiges \(\ast\)-Ideal in \( \mathcal{A} \). Der Quotient \( \mathcal{A}/I \) ist definiert als die Menge der Äquivalenzklassen \( \{[a] \mid [a] = a + I \} \) für \( a \in \mathcal{A} \), mit
\begin{align}
    [a] + [b] = [a + b], \quad [a] \cdot [b] = [a b], \quad [a]^* = [a^*],
\end{align}
und der Norm $\| [a] \| = \inf \{ \| a + i \| \mid i \in I \}$. Mit dieser Norm wird \( \mathcal{A}/I \) zu einer \( C^\ast \)-Algebra.
\end{proposition}

\begin{proof}
    Einfaches nachrechnen.
\end{proof}

\begin{theorem}
Die Menge $\mathcal{K}(\h)$ ist ein abgeschlossenes beidseitiges $\ast$-Ideal in $\mathcal{B}(\h)$. Sei $\iota: \mathcal{K}(\h) \to \mathcal{B}(\h)$ die Einbettung. Dann ist der Quotient $\mathcal{Q}(\h) \coloneq \mathcal{B}(\h)/\mathcal{K}(\h)$ eine $C^\ast$-algebra, auch genannt die Calkin-Algebra. Zusammen mit $\mathcal{K}(\h)$ und $\mathcal{B}(\h)$ bildet diese eine kurze exakte Sequenz von $C^\ast$-algebren
\begin{equation}
    0 \to \mathcal{K}(\h) \xrightarrow[]{\iota} \mathcal{B}(\h) \xrightarrow[]{\pi} \mathcal{Q}(\h) \to 0,
\end{equation}
welche auch Calkin exakte Sequenz genannt wird. Die Projektion $\pi$ auf den Quotienten $\mathcal{Q}(\h)$ heißt Calkin-Projektion.
\end{theorem}

\begin{proof}
Wir zeigen als erstes, dass für alle $T \in \mathcal{K}(\mathcal{H})$ und $S \in \mathcal{B}(\mathcal{H})$ sowohl $ST$ als auch $TS$ in $\mathcal{K}(\mathcal{H})$ liegen. Zuerst betrachten wir das Produkt $ST$.

Sei $\mathcal{B} \subseteq \mathcal{H}$ eine beschränkte Menge. Da $T$ kompakt ist, ist das Bild $T(\mathcal{B})$ kompakt, da kompakte Operatoren beschränkte Mengen auf Mengen mit kompaktem Abschluss abbilden. Da $S$ ein beschränkter Operator ist, ist $S(T(\mathcal{B}))$ ebenfalls beschränkt. Da $T(\mathcal{B})$ kompakt ist, bleibt $S(T(\mathcal{B}))$ kompakt, weil das Bild unter einem beschränkten Operator von einer kompakten Menge kompakt bleibt. Folglich ist $ST \in \mathcal{K}(\mathcal{H})$.

Nun betrachten wir das Produkt $TS$. Da $S$ beschränkt ist, ist $S(\mathcal{B})$ ebenfalls beschränkt. Da $T$ kompakt ist, ist $T(S(\mathcal{B}))$ kompakt. Folglich ist $TS \in \mathcal{K}(\mathcal{H})$. Also ist $\mathcal{K}(\mathcal{H})$ ein Ideal in $\mathcal{B}(\mathcal{H})$.

Da wir in Satz \ref{schauder} bereits gezeigt haben, dass auch $T^\ast, S^\ast \in \mathcal{K}(\mathcal{H})$, können wir ein analoges Argument verwenden um zu zeigen, dass auch $T^\ast S^\ast, S^\ast T^\ast \in \mathcal{K}(\mathcal{H})$. Also ist $\mathcal{K}(\mathcal{H})$ ein $\ast$-Ideal in $\mathcal{B}(\mathcal{H})$.

Die Menge $\mathcal{K}(\mathcal{H})$ ist abgeschlossen in der normierten Struktur von $\mathcal{B}(\mathcal{H})$, da jeder konvergente Folge von kompakten Operatoren gegen einen kompakten Operator konvergiert.\footnote{Sei \((K_n)_{n\geq 1} \subset \mathcal{K}(\mathcal{H})\) eine konvergente Folge kompakter Operatoren in der Operatornorm, d.h., es existiert ein \(T \in \mathcal{B}(\mathcal{H})\), sodass $\|K_n - T\| \to 0 \quad \text{für} \quad n \to \infty$. Zu zeigen ist, dass \(T \in \mathcal{K}(\mathcal{H})\). Sei \((x_n)_{n\geq 1}\) eine beschränkte Folge in \(\mathcal{H}\), d.h. es existiert ein \(C > 0\), sodass \(\|x_n\| \leq C\) für alle \(n \in \mathbb{N}\). Da jeder \(K_n\) kompakt ist, existiert für jede Folge \((x_n)_{n\geq 1}\) eine Teilfolge \((x_{n_k})_{k\geq 1}\), sodass \(K_n(x_{n_k})_{k\geq 1}\) eine konvergente Teilfolge in \(\mathcal{H}\) bildet. Wir müssen nun zeigen, dass \(T(x_n)_{n\geq 1}\) ebenfalls eine konvergente Teilfolge besitzt. Weil \(\|K_n - T\| \to 0\), folgt für jede beschränkte Folge \((x_k)_{k\geq 1}\), dass $\|K_n(x_k)_{k\geq 1} - T(x_k)_{k\geq 1}\|_\h \leq \|K_n - T\| \|(x_k)_{k\geq 1}\|_\h \to 0 \quad \text{für} \quad n \to \infty$. Da \(K_n(x_{k_l})_{l\geq 1}\) eine konvergente Teilfolge ist und \(T(x_k)_{k\geq 1}\) nahe bei \(K_n(x_k)_{k\geq 1}\) liegt, folgt, dass auch \(T(x_k)_{k\geq 1}\) eine konvergente Teilfolge besitzt. Somit ist \(T\) kompakt. Also ist \(T \in \mathcal{K}(\mathcal{H})\), und daher ist \(\mathcal{K}(\mathcal{H})\) abgeschlossen in \(\mathcal{B}(\mathcal{H})\).}

Wir definieren die Einbettung $\iota: \mathcal{K}(\mathcal{H}) \to \mathcal{B}(\mathcal{H})$ durch $\iota(T) = T$ für $T \in \mathcal{K}(\mathcal{H})$. Da $\iota$ eine Inklusion ist, ist sie injektiv. Die Menge $\mathcal{Q}(\mathcal{H})$ wird als der Quotient $\mathcal{B}(\mathcal{H}) / \mathcal{K}(\mathcal{H})$ definiert. Zwei Operatoren $T, S \in \mathcal{B}(\mathcal{H})$ sind äquivalent modulo $\mathcal{K}(\mathcal{H})$, wenn $T - S \in \mathcal{K}(\mathcal{H})$. Die Menge $\mathcal{Q}(\mathcal{H})$ ist mit den Operationen der Addition und Skalarmultiplikation, sowie der Multiplikation der Äquivalenzklassen und der $\ast$-Operation ausgestattet:
   \[
   [T] + [S] = [T + S], \quad \lambda[T] = [\lambda T], \quad [T][S] = [TS], \quad [T]^\ast = [T^\ast].
   \]
Diese Operationen sind wohldefiniert und erfüllen die $C^\ast$-Algebra-Axiome.

Wir haben eine kurze Sequenz von $C^\ast$-Algebren:
   \[
   0 \to \mathcal{K}(\mathcal{H}) \xrightarrow[]{\iota} \mathcal{B}(\mathcal{H}) \xrightarrow[]{\pi} \mathcal{Q}(\mathcal{H}) \to 0.
   \]
Hier ist $\pi$ die Projektion auf den Quotienten $\mathcal{Q}(\mathcal{H})$, und diese Sequenz ist genau dann exakt, wenn das Bild von $\iota$ gleich dem Kern von $\pi$ ist. Es gilt $\ker(\pi) = \mathcal{K}(\h) = \iota(\mathcal{K}(\h))$.
\end{proof}

\begin{theorem}[Satz von Riesz-Schauder]
Für $K \in \mathcal{K}(\h)$ definieren wir $T = \mathbf{1}-K$. Dann gelten die folgenden Aussagen:
\begin{enumerate}
    \item Es gibt ein $n \geq 1$, so dass $\ker(T^k) = \ker(T^n)$ für alle $k \geq n$.
    \item $\im(T) = T(\h)$ ist ein abgeschlossener Unterraum.
    \item $\dim(\ker(T)) = \dim(\ker(T^\ast)) < \infty$.
\end{enumerate}
\end{theorem}

\begin{proof}
Nachzulesen in \cite[Satz VI.2.1, Lemma VI.2.2]{werner2018funktionalanalysis}.
\end{proof}

\begin{definition}
    Das Spektrum $\spec(T)$ eines beschränkten linearen Operators besteht aus allen Punkten $\lambda \in \mathbb{C}$ für die $\lambda \mathbf{1} - T$ nicht invertierbar ist. Das Punktspektrum $\spec_p(T)$ von $T$ besteht aus allen Eigenwerten von $T$, nämlich die $\lambda \in \mathbb{C}$, für die $\ker(\lambda \mathbf{1} - T) \neq \{0\}$.
\end{definition}

\begin{theorem}[Rieszsche Spektraltheorie kompakter Operatoren]
\label{Riesz}
Das Spektrum $\spec(K)$ von jedem kompakten Operator $K \in \mathcal{K}(\h)$ ist eine abzählbare Menge $\{\lambda_j \; \vert \; j \geq 1\} \cup \{0\}$, wobei alle $\lambda_j \neq 0$ Eigenwerte mit endlicher Multiplizität sind, welche als einzigen Häufungspunkt die $0$ haben. Weiterhin kann $0$ ein Eigenwert von endlicher oder unendlicher Multiplizität sein. Ist $0$ ein Eigenwert von endlicher Multiplizität, so ist $0$ ein Häufungspunkt der Folge $(\lambda_j)_{j \geq 1}$.
\end{theorem}

\begin{proof}
Nachzulesen in \cite[VI.2.5]{werner2018funktionalanalysis}
\end{proof}

\section{Eigenschaften beschränkter Fremdholmoperatoren}

\begin{theorem}
    \label{fredholm}
    Für $T \in \mathcal{B}(\h,\mathcal{Y})$ sind die folgenden Aussagen äquivalent:
    \begin{enumerate}
        \item $T$ ist ein Fredholmoperator,
        \item $ \dim(\ker(T)) < \infty$ und $\dim(\mathcal{Y} / \im(T)) < \infty$,
        \item es gibt ein eindeutiges $T^\dagger_0 \in \mathcal{B}(\mathcal{Y},\h)$ mit
        \begin{equation}
            \ker(T^\dagger_0) = \ker(T^\ast), \quad \ker(T^{\dagger\ast}_0) = \ker(T),
        \end{equation}
        so dass $T^\dagger_0T$ und $TT^\dagger_0$ orthogonale Projektionen auf $\ker(T)^\perp$ und $\ker(T^\ast)^\perp$ sind und
        \begin{equation}
            \dim(\im(\mathbf{1}-T^\dagger_0T)) < \infty, \quad \dim(\im(\mathbf{1}-TT^\dagger_0)) < \infty.
        \end{equation}
        \item Es gibt einen Operator $T^\dagger\in \mathcal{B}(\mathcal{Y},\h)$ für $T$, welcher die Bedingung $TT^\dagger-\mathbf{1} \in \mathcal{K}(\mathcal{Y})$ und $T^\dagger T-\mathbf{1} \in \mathcal{K}(\h)$ erfüllt. Dieser heißt im Folgenden Pseudoinverses.
    \end{enumerate}
\end{theorem}

\begin{proof}\noindent
\begin{itemize}
    \item 1 $\implies$ 2: Da $T$ Fredholmsch ist, folgt aus der Definition $\dim(\ker(T)) < \infty$. Da nach \cite[Lemma 2.23]{lechner} gilt $\ker(T^\ast) = \im(T)^\perp$, ist $\dim(\ker(T^\ast)) = \dim(\im(T)^\perp) < \infty$ und $\im(T)^\perp$ ist abgeschlossen \cite[Lemma 2.4]{lechner}. Es gilt \( \mathcal{Y} = \im(T) \oplus \im(T)^\perp \) aufgrund des abgeschlossenen Bilds von $T$. Weiterhin kann jedes \( \phi \in \mathcal{Y} \) eindeutig als \( \phi = \hat{\phi} + \hat{\phi}^\perp \) geschrieben werden, wobei \( \hat{\phi} \in \im(T) \) und \( \hat{\phi}^\perp \in \im(T)^\perp \) ist. Da $\im(T)$ abgeschlosser Unterraum eines Hilbertraums ist, ist dieser selbst ein Hilbertraum \cite[Satz 4.7]{werner2018funktionalanalysis}. Selbiges gilt für dessen orthogonales Komplement $\im(T)^\perp$. Damit können wir den Hilbertraumquotienten $\mathcal{Y} / \im(T)$ bilden. Definiere die Abbildung \( \Phi: \mathcal{Y} / \im(T) \to \im(T)^\perp \) durch \( \Phi([\phi]) = \hat{\phi}^\perp \), wobei \( \hat{\phi}^\perp \) der orthogonale Anteil von \( \phi \) ist. Diese Abbildung ist wohldefiniert, da, wenn \( \phi_1 \sim \phi_2 \), \( \phi_1 - \phi_2 \in \im(T) \) gilt, also die orthogonalen Komponenten gleich sind. \( \Phi \) ist injektiv, da aus \( \Phi([\phi]) = 0 \) folgt, dass \( \hat{\phi}^\perp = 0 \) ist, also \( \phi \in \im(T) \), was \( [\phi] = [0] \) impliziert. Surjektivität folgt, da für jedes \( \hat{\phi}^\perp \in \im(T)^\perp \) gilt, dass \( \Phi([\hat{\phi}^\perp]) = \hat{\phi}^\perp \) ist. Daher ist \( \Phi \) ein Isomorphismus (die Linearität der Abbildung ergibt sich aus der Definition), und es folgt, dass \( \mathcal{Y} / \im(T) \cong \im(T)^\perp \) gilt. Also haben wir $\dim(\mathcal{Y}/\im(T)) < \infty$.
    \item 2 $\implies$ 1: Da $\ker(T^\ast) = \im(T)^\perp$, bleibt es zu zeigen, dass $\dim(\mathcal{Y}/\im(T))) < \infty$ impliziert, dass $\im(T)$ abgeschlossen ist. Dazu betrachten wir die Einschränkung $\widetilde{T} \coloneq T\vert_{\ker(T)^\perp}: \ker(T)^\perp \rightarrow \mathcal{Y}$. Diese ist stetig, injektiv und hat Bild $\im(\widetilde{T}) = \im(T)$. Es genügt also, die Aussage für eine injektive Abbildung mit endlich-dimensionalem Kokern zu zeigen, welche wir ebenfalls mit $T$ bezeichnen. Sei $\{\phi_1, \ldots, \phi_N\}$ eine Basis von $\mathcal{Y}/\im(T)$. Dann können wir eine lineare Abbildung $\hat{T}: \mathbb{C}^N \oplus \h \to \mathcal{Y}$ durch
    \begin{align}
        \hat{T}(\lambda_1, \ldots, \lambda_N, \psi) = \sum_{n=1}^{N} \lambda_n \phi_n + T\psi
    \end{align}
    definieren. Diese Abbildung ist bijektiv und stetig. Also impliziert der Satz vom stetigen Inversen,\footnote{Dieser impliziert, dass für zwei Hilberträume $\h,\mathcal{Y}$ und $T \in \mathcal{B}(\h,\mathcal{Y})$ bijektiv, die Umkehrabbildung $T^{-1} \in \mathcal{B}(\mathcal{Y},\h)$ ebenfalls beschränkt ist \cite[Korollar 3.24]{lechner}.} dass auch $\hat{T}^{-1}$ stetig ist. Da $0 \oplus \h \subset \mathbb{C}^N \oplus \h$ abgeschlossen ist, weil $\h$ abgeschlossen ist, folgt auch, dass dessen Urbild unter einer stetigen Abbildung abgeschlossen ist. Also ist $\im(T) = \hat{T}(0,\h) = (\hat{T}^{-1})^{-1}(0,\h)$ abgeschlossen.
    \item 1 $\implies$ 3: Da $T\vert_{\ker(T)^\perp}: \ker(T)^\perp \to \im(T)$ nach Annahme eine bijektive stetige lineare Abbildung zwischen Hilberträumen ist, also insbesondere Banachräumen, impliziert der Satz vom stetigen Inversen \cite[Korollar 3.24]{lechner} die Existenz einer stetigen Inversen Abbildung $T^\dagger_0: \im(T) \to \ker(T)^\perp$ ($T^\dagger_0$ kann verstanden werden als die Einschränkungs $T^\ast\vert_{\im(T)}$). Diese Abbildung kann auf ganz $\mathcal{Y}$ fortgesetzt werden, indem wir $T^\dagger_0 \psi = 0$ für $\psi \in \im(T)^\perp$ setzen. Dann ergibt sich $(TT^\dagger_0)^2 = TT^\dagger_0$ und $(TT^\dagger_0)^\ast = T^{\dagger\ast}_0 T^\ast =  TT^\dagger_0$ (dies sieht man, da die $TT^\dagger_0\vert_{\im(T)} = \mathbf{1}_{\im(T)}$ und $T^\dagger_0T\vert_{\ker(T)^\perp} = \mathbf{1}_{\ker(T)^\perp}$, während $TT^\dagger_0\vert_{\im(T)^\perp} = 0$ und $T^\dagger_0\vert_{\im(T)^\perp}T = 0$), also
    \begin{align*}
        TT^\dagger_0  \text{ ist orthogonale Projektion in $\mathcal{Y}$ auf } \im(T) &= \overline{\im(T)} = \ker(T^\ast)^\perp, \\
        T^\dagger_0 T \text{ ist orthogonale Projektion in $\h$ auf } \ker(T)^\perp &= \overline{\im(T^\ast)}.
    \end{align*}
    Die Eindeutigkeit der linearen Abbildung folgt sofort. Es gilt $\ker(T_0^\dagger) = \im(T)^\perp = \ker(T^\ast)$ und somit auch $\ker(T_0^{\dagger\ast}) = \im(T^\ast)^\perp = \ker(T)$. Die Operatoren $1-TT^\dagger_0$ und $1-T^\dagger_0 T$ wirken auf $\im(T)^\perp = \ker(T^\ast)$ bzw. $\ker(T)^{\perp\perp} = \ker(T)$, welche endlichdimensional sind, da $T$ Fredholmsch ist. Also sind deren Bilder auch endlichdimensional.
    \item 3 $\implies$ 4: Das folgt sofort aus der Tatsache, dass jeder beschränkte lineare Operator mit endlich-dimensionalem Bild kompakt ist -- unter Verwendung von Heine-Borel - da die Bilder Teilmengen von abgeschlossenen und beschränkten Mengen in einem endlichdimensionalen $\mathbb{C}$-Vektorraum sind.
    \item 4 $\implies$ 1: Wir nehmen an, dass $(\psi_n)_{n \geq 1}$ eine unendliche Orthonormalbasis von $\ker(T)$ ist. Diese Vektoren sind alle Eigenvektoren des kompakten Operators $K = T^\dagger T -1$ zum Eigenwert $-1$. Dies ist ein Widerspruch zu Satz \ref{Riesz}. Damit ist $\dim(\ker(T)) < \infty$.

    Für $\ker(T^\ast)$ können wir auf dieselbe Art und Weise argumentieren, indem wir den kompakten Operator $\widetilde{K} = (TT^\dagger -1)^\ast$ verwenden. Also ist auch $\dim(\ker(T^\ast)) < \infty$.

    Es bleibt zu zeigen, dass $\im(T)$ abgeschlossen ist. Sei $K = T^\dagger T - 1$ und $L \in \mathcal{K}(\h)$ mit endlich dimensionalem Bild, so dass
    \begin{align}
        \Vert K-L \Vert_\h \leq \frac{1}{2}.
    \end{align}
    Dann gilt für alle $\phi \in \ker(L)$:
    \begin{align}
        \Vert T^\dagger \Vert_\h \Vert T\phi \Vert_\h &\geq \Vert T^\dagger T \phi \Vert_\h = \Vert (\mathbf{1}+K)\phi \Vert_\h \nonumber\\
        &= \Vert \phi + K\phi \Vert_\h \nonumber\\
        &\geq \Vert \phi \Vert_\h - \Vert K \phi \Vert_\h \nonumber\\
        &= \Vert \phi \Vert_\h - \Vert (K-L+L) \phi \Vert_\h \nonumber\\
        &\geq \Vert \phi \Vert_\h - (\Vert (K-L) \phi \Vert_\h + \Vert L \phi \Vert_\h) \nonumber\\
        &= \Vert \phi \Vert_\h - \Vert (K-L)\phi \Vert_\h - 0 \nonumber\\
        &\geq \Vert \phi \Vert_\h - \frac{1}{2} \Vert \phi \Vert_\h \nonumber\\
        &= \frac{1}{2} \Vert \phi \Vert_\h.
    \end{align}
    Also gilt $\Vert \phi \Vert_\h \leq 2 \Vert T^\dagger \Vert_\h \Vert T\phi \Vert_\h$ für alle $\phi \in \ker(L)$. Falls nun $(T\phi_n)_{n \geq 1}$ eine Folge ist, mit $\phi_n \in \ker(L)$ und $\psi = \lim_{n \to \infty} T\phi_n$, dann gilt
    \begin{align}
        \Vert \phi_n - \phi_m \Vert_\h \leq 2 \Vert T^\dagger \Vert_\h \Vert T\phi_n - T\phi_m \Vert_\h \xrightarrow[]{n,m \to \infty} 0.
    \end{align}
    Also ist $(\phi_n)_{n \geq 1}$ eine Cauchy-Folge und hat Grenzwert $\phi = \lim_{n \to \infty} \phi_n \in \ker(L)$, wobei wir verwendet haben, dass $\overline{\ker(L)} = \ker(L)$ abgeschlossen ist, da der Kern eines jeden stetigen linearen Operators abgeschlossen ist.\footnote{$\ker(T) = T^{-1}(\{0\})$ und $\{0\} \subset \mathcal{Y}$ ist eine abgeschlossene Menge. Demnach ist das Urbild einer abgeschlossenen Menge unter einer stetigen Abbildung abgeschlossen, also ist $\ker(T)$ abgeschlossen.} Aufgrund der Stetigkeit von $T$ folgt auch $\psi = T\phi \in T(\ker(L))$. Andererseits
    \begin{align}
        T(\ker(L)^\perp) = T(\im(L^\ast)),
    \end{align}
    wobei $\im(L^\ast)$\footnote{Sei $\im(L) = \operatorname{span}_\mathbb{C}\{\varphi_1, \dots, \varphi_n\}$. Für jedes \( \varphi^\ast \in \im(L^\ast) \) gilt dann $\varphi^\ast = \sum_{n=1}^N \alpha_n \varphi_n^\ast, \text{wobei } \varphi_n^\ast(\varphi_m) = \delta_{nm}$. Also wirkt der adjungierte Operator als $L^\ast \varphi^\ast = \sum_{n=1}^N \alpha_n L^\ast \varphi_n^\ast$. Damit ergibt sich für das Bild \( \im(L^\ast) = \operatorname{span}_{\mathbb{C}}\{L^\ast \varphi_1^\ast, \dots, L^\ast \varphi_n^\ast\} \).} endliche Dimension hat und somit abgeschlossen ist. Also ist auch $T(\ker(L)^\perp)$ endlichdimensional. Damit ist $\im(T) = T(\ker(L)) + T(\ker(L)^\perp)$ abgeschlossen als Summe abgeschlossener, endlichdimensionaler Unterräume.
\end{itemize}
\end{proof}

\begin{remark}
Für die Definition eines Fredholmoperators reicht es nicht zu fordern, dass die Dimension des Kerns und Kokerns endlich sein soll. Wir können die Bedingung nicht fallen lassen, dass der Operator ein abgeschlossenes Bild haben soll, wie Satz \ref{fredholm}.2 fälschlicherweise suggeriert. Betrachten wir dazu den selbstadjungierten Operator $T: \ell^2(\mathbb{N}) \to \ell^2(\mathbb{N}), Tx \coloneq \frac{1}{n} (x_n)_{n \geq 1}$. Es gilt $\ker(T) = \ker(T^\ast) = \{0\}$ und $\dim(ker(T)) = \dim(\ker(T^\ast)) < \infty$. Wir zeigen, dass $T$ kompakt ist. Betrachte $$\mathbb{B}_{\ell^2(\mathbb{N})} := \left\{ x \in \ell^2(\mathbb{N}) \; \bigg\vert \; \Vert x \Vert_2 \coloneq \left(\sum_{n=1}^{\infty} \vert x_n \vert^2\right)^{\frac{1}{2}} \leq 1 \right\},$$ die Einheitskugel in $\ell^2(\mathbb{N})$. Dann ist für $x \in \mathbb{B}_{\ell^2(\mathbb{N})}$ die Norm $\Vert Tx \Vert_2 = (\sum_{n=1}^\infty \frac{1}{n^2} \vert x_n \vert^2)^\frac{1}{2} \leq (\sum_{n=1}^\infty \vert x_n \vert^2)^\frac{1}{2} \leq 1 < \infty$ beschränkt. Also ist das Bild präkompakt und $T$ somit kompakt. Betrachte die Folge $y = (\frac{1}{n})_{n \geq 1}$. Angenommen es gäbe ein $x \in \ell^2(\mathbb{N})$, so dass $Tx = y$, dann müsste gelten $\frac{1}{n} x_n = \frac{1}{n}$, also $x_n = 1$ für alle $n \geq 1$. Dies führt zu $\Vert x \Vert_2 = \left( \sum_{n=1}^{\infty} 1 \right)^\frac{1}{2} = \infty$, also wäre $x \notin \ell^2(\mathbb{N})$ und damit ist $y \notin \im(T)$. Betrachte die Folge $\widetilde{y} := \frac{1}{n} x_n$, wobei \( x_n = \chi_{\{1, \ldots, n\}}(\{n\})\). Dann ist $x \in \ell^2(\mathbb{N})$ da \( \Vert x \Vert^2 = \sum_{n \geq 1} \vert x_n \vert^2 = \sum_{n=1}^{N} \vert x_n \vert^2 < \infty\) für alle \( N \in \mathbb{N} \). Insbesondere $\widetilde{y} \in \im(T)$. Außerdem gilt $\Vert y - \widetilde{y} \Vert_2 \xrightarrow[]{n \to \infty} 0$ also $\widetilde{y} \to y$. Also ist $\im(T)$ nicht abgeschlossen.
\end{remark}

Als Konsequenzen ergeben sich folgende beiden Korollare:

\begin{corollary}\noindent
    \begin{enumerate}
        \item Für $T \in \mathcal{FB}(\h,\mathcal{Y})$, $T' \in \mathcal{FB}(\mathcal{Z},\h)$, ist auch $TT'\in \mathcal{FB}(\mathcal{Z},\mathcal{Y})$.
        \item Falls $T \in \mathcal{FB}(\h,\mathcal{Y})$, dann ist $T^\ast \in \mathcal{FB}(\mathcal{Y},\h)$ und $\ind(T^\ast) = - \ind(T)$.
        \item Falls $A \in \mathcal{B}(\h,\mathcal{Y})$ invertierbar ist, dann ist $A \in \mathcal{FB}(\h,\mathcal{Y})$ und $\ind(A) = 0$.
        \item Für $T \in \mathcal{FB}(\h,\mathcal{Y})$ und invertierbare Operatoren $A \in \mathcal{B}(\mathcal{Y},\mathcal{Z})$ und $B \in \mathcal{B}(\mathcal{Z},\h)$, erhalten wir für die Indizes $\ind(AT) = \ind(TB) = \ind(T)$.
        \item Für $T \in \mathcal{FB}(\h,\mathcal{Y})$ gilt
        \begin{align}
            \ind(T) &= \dim(\ker(T^\ast T)) - \dim(\ker(TT^\ast)) \\
                    &= \dim(\ker(T)) - \dim(\ker(T^\ast)).
        \end{align}
        Diese Gleichheit ergibt sich aus $\Vert T\varphi \Vert^2 = \langle T\varphi, T\varphi \rangle_\h = \langle \varphi, T^\ast T\varphi \rangle_\h = 0$, dann ist $\varphi \in \ker(T)$ aufgrund der Norm.
        \item Für $T \in \mathcal{FB}(\h,\mathcal{Y})$ und $T' \in \mathcal{FB}(\mathcal{Z},\mathcal{Z}')$ erhalten wir $T \oplus T' \in \mathcal{FB}(\h \oplus \mathcal{Z}, \mathcal{Y} \oplus \mathcal{Z}')$ und $\ind(T \oplus T') = \ind(T) + \ind(T')$.
    \end{enumerate}
\end{corollary}

\begin{remark}
Um das vorherige Korollar kurz zu erklären, hier meine Bemerkungen:
    \begin{enumerate}
        \item Es gilt $\ker(TT') \subset \ker(T')$ und $\ker((TT')^\ast) = \ker(T'^\ast T^\ast) \subset \ker(T^\ast)$. Da $T'$ Fredholm ist, gilt auch $\dim(\ker(TT')) < \infty$ und $\dim(\ker((TT')^\ast) < \infty$. Die Abgeschlossenheit des Bilds folgt trivial.
        \item Da $ker(T^{\ast\ast}) = \ker(T)$ folgt die Aussage sofort.
        \item Ist $A$ invertierbar, so ist $\ker(A) = \{0\}$, also $\dim(\ker(A)) < \infty$. Außerdem ist $\im(A) = \mathcal{Y}$ und somit $\coker(A) = \{0\}$ und damit $\dim(\coker(A)) < \infty$. Es gilt damit $\ind(A) = 0 - 0 = 0$.
        \item Für $AT$ gilt $\ker(AT) = \{ x \in \mathcal{H} \mid ATx = 0 \} = \{ x \in \mathcal{H} \mid Tx = 0 \} = \ker(T)$, also $\dim(\ker(AT)) = \dim(\ker(T))$. Da $A$ invertierbar ist, induziert $A$ einen Isomorphismus von $\operatorname{im}(T)$ auf $\operatorname{im}(AT)$ und somit ist $\operatorname{coker}(AT) \cong \operatorname{coker}(T)$, was $\dim(\operatorname{coker}(AT)) = \dim(\operatorname{coker}(T))$ impliziert. Daraus folgt $\operatorname{ind}(AT) = \dim(\ker(AT)) - \dim(\operatorname{coker}(AT)) = \dim(\ker(T)) - \dim(\operatorname{coker}(T)) = \operatorname{ind}(T)$. Für $TB$ gilt analog $\ker(TB) \cong \ker(T)$ und $\operatorname{coker}(TB) \cong \operatorname{coker}(T)$, also $\dim(\ker(TB)) = \dim(\ker(T))$ und $\dim(\operatorname{coker}(TB)) = \dim(\operatorname{coker}(T))$. Daraus folgt $\operatorname{ind}(TB) = \dim(\ker(TB)) - \dim(\operatorname{coker}(TB)) = \operatorname{ind}(T)$. Somit ist $\operatorname{ind}(AT) = \operatorname{ind}(TB) = \operatorname{ind}(T)$.
        \item Hier ist nichts zu zeigen.
        \item Es ist $\ker(T \oplus T') = \ker(T) \oplus \ker(T')$, also $\dim(\ker(T \oplus T')) = \dim(\ker(T)) + \dim(\ker(T'))$. Zudem ist $\operatorname{im}(T \oplus T') = \operatorname{im}(T) \oplus \operatorname{im}(T')$, und da die Bilder von $T$ und $T'$ abgeschlossen sind, ist auch $\operatorname{im}(T \oplus T')$ abgeschlossen. Der Kokern ist $\operatorname{coker}(T \oplus T') = (\mathcal{Y} \oplus \mathcal{Z}') / \operatorname{im}(T \oplus T') \cong \operatorname{coker}(T) \oplus \operatorname{coker}(T')$, also $\dim(\operatorname{coker}(T \oplus T')) = \dim(\operatorname{coker}(T)) + \dim(\operatorname{coker}(T'))$. Somit folgt
        \begin{align}
        \operatorname{ind}(T \oplus T') &= \dim(\ker(T \oplus T')) - \dim(\operatorname{coker}(T \oplus T')) \nonumber\\
        &= [\dim(\ker(T)) + \dim(\ker(T'))] - [\dim(\operatorname{coker}(T)) + \dim(\operatorname{coker}(T'))] \nonumber\\
        &= \operatorname{ind}(T) + \operatorname{ind}(T').
        \end{align}
    \end{enumerate}
\end{remark}

\begin{corollary}
    \label{summeFredholm}
    Falls $T \in \mathcal{B}(\h,\mathcal{Y})$ ein Fredholmoperator ist und $K \in \mathcal{K}(\h,\mathcal{Y})$ kompakt, dann ist $T+K$ auch ein Fredholmoperator.
\end{corollary}

\begin{proof}
Sei $T^\dagger$ das Pseudoinverse von $T$. Dann ist $TT^\dagger - \mathbf{1} \in \mathcal{K}(\mathcal{Y})$ und $T^\dagger T - \mathbf{1} \in \mathcal{K}(\h)$. Aber dann gilt für $(T+K)T^\dagger - \mathbf{1} = TT^\dagger + KT^\dagger - \mathbf{1} \in \mathcal{K}(\mathcal{Y})$ und $T^\dagger (T+K) -\mathbf{1} = T^\dagger T+ T^\dagger K -\mathbf{1}  \in \mathcal{K}(\h)$. Also ist $T^\dagger$ auch Pseudoinverses von $T+K$ und letzterer Operator nach Satz \ref{fredholm}.4 Fredholmsch.
\end{proof}


\begin{theorem}
    Ein Operator $T \in \mathcal{B}(\h)$ ist Fredholmsch genau dann, wenn das Bild $\pi(T)$ von $T$ in der Calkin-Algebra invertierbar ist.
\end{theorem}

\begin{proof}
    Sei $T$ ein Fredholmoperator. Nach Satz \ref{fredholm}.4 gibt es einen Operator $T^\dagger \in \mathcal{B}(\h)$, so dass $T^\dagger T - 1, TT^\dagger - 1 \in \mathcal{K}(\h)$. Da $\pi$ als kanonische Surjektion ein Algebrahomomorphismus ist und $\pi(K) = [0]$ für alle $K \in \mathcal{K}(\h)$, gilt
    \begin{align}
        [0] &=  \pi(TT^\dagger - \mathbf{1}) = \pi(T)\pi(T^\dagger) - \pi(\mathbf{1}) = \pi(T)\pi(T^\dagger)-[\mathbf{1}], \\
        [0] &= \pi(T^\dagger)\pi(T)-\pi(\mathbf{1}) = \pi(T^\dagger)\pi(T)-[\mathbf{1}].
    \end{align}
    Also ist $\pi(T)$ invertierbar mit Inverser $\pi(T^\dagger)$.

    Andererseits ist $\pi$ eine Surjektion. Sei $[T] = \pi(T) \in \mathcal{Q}(\h)$ invertierbar mit Inverser $[T^\dagger]$, so dass
    \begin{align}
        [T][T^\dagger] - [\mathbf{1}] = [0] =[T^\dagger][T] - [\mathbf{1}].
    \end{align}
    Da $\pi$ surjektiv ist, gibt es ein $T^\dagger \in \mathcal{B}(\h)$, so dass $\pi(T^\dagger) = [T^\dagger]$. Weil $\pi$ auch ein Homomorphismus ist, folgt daraus
    \begin{align}
        \pi(TT^\dagger - \mathbf{1}) = \pi(T^\dagger T -\mathbf{1}) = [0].
    \end{align}
    Folglich ist $TT^\dagger-\mathbf{1}, T^\dagger T -\mathbf{1} \in \mathcal{K}(\h)$ und $T$ ist ein Fredholmoperator nach Satz \ref{fredholm}.4.
\end{proof}

Wir führen nun ein erstes Kriterium ein, mit dem wir entscheiden können, ob ein gegebener Operator Fredholmsch ist oder nicht. Es gibt insgesamt zwei weit verbreitete Kriterien um dies zu prüfen, das zweite Kriterium ist in \cite[Theorem 3.4.1]{doll2023spectral} nachzulesen.

\begin{proposition}
    Sei $T \in \mathcal{B}(\h,\mathcal{Y})$ ein beschränkter linearer Operator. Falls es kompakte lineare Operatoren $K \in \mathcal{K}(\h,\mathcal{Z})$ gibt und eine Konstante $c > 0$, wobei $\mathcal{Z}$ ein weiterer separabler Hilbertraum ist, so dass die folgende Bedingung
    \begin{align}
        \Vert\phi\Vert_\h \leq c (\Vert T\phi \Vert_{\mathcal{Y}} + \Vert K\phi \Vert_{\mathcal{Z}})
    \end{align}
    für alle $\phi \in \h$ erfüllt ist, dann ist $\im(T)$ abgeschlossen und $T$ hat endlichdimensionalen Kern.
\end{proposition}

\begin{proof}
    Sei $(\phi_n)_{n \geq 1}$ eine beschränkte Folge in $\h$, sodass $T \phi_n$ konvergent ist in $\mathcal{Y}$, es soll also ein $\psi \in \mathcal{Y}$ geben, mit $\lim_{n\to\infty} T\phi_n = \psi$. Weil $K$ kompakt ist, gibt es eine Teilfolge $(\phi_{n_k})_{k \geq 1}$, so dass $K \phi_{n_k}$ konvergent ist. Dann ist $(K\phi_{n_k})_{k \geq 1}$ eine Cauchy-Folge und $\lim_{k\to\infty} T\phi_{n_k} = \psi$, also ist auch $(T\phi_{n_k})_{k \geq 1}$ eine Cauchy-Folge. Deshalb gibt es für alle $\epsilon > 0$ ein $N\in \mathbb{N}$, so dass
    \begin{align}
        \max(\Vert T\phi_{n_k}-T\phi_{n_m} \Vert_{\mathcal{Y}}, \Vert K\phi_{n_k}-K\phi_{n_m} \Vert_{\mathcal{Z}}) < \frac{\epsilon}{2c}
    \end{align}
    für alle $k,m > N$. Daraus folgt
    \begin{align}
        \Vert \phi_{n_k} - \phi_{n_m} \Vert_\h \leq c (\Vert T\phi_{n_k}-T\phi_{n_m} \Vert_{\mathcal{Y}} +  \Vert K\phi_{n_k}-K\phi_{n_m} \Vert_{\mathcal{Z}}) < \epsilon
    \end{align}
    für alle $k,m > N$. Also ist $(\phi_{n_k})_{k \geq 1}$ eine Cauchy-Folge in einem Hilbertraum und somit konvergent.

    Wir nehmen an, dass $\dim(\ker(T))$ unendlichdimensional ist und $\{\phi_n \;\vert\; n \geq 1\}$ eine Orthonormalbasis des Kerns von $T$ ist. Dann ist $(\phi_n)_{n \geq 1}$ eine beschränkte Folge in $\h$, so dass $T \phi_n$ konstant $0$ ist und damit ebenfalls konvergent. Dies ist ein Widerspruch, da $(\phi_n)_{k \geq 1}$ keine konvergente Teilfolge besitzt. Also ist $\dim(\ker(T)) < \infty$.

    Ferner gibt es eine Konstante $c_1 > 0$, so dass $\Vert \psi \Vert_\h \leq c_1 \Vert T\psi \Vert_{\mathcal{Y}}$ für alle $\psi \in \ker(T)^\perp$, denn sonst gäbe es eine Folge $(\psi_n)_{k \geq 1}$ in $\ker(T)^\perp$ mit $\Vert \psi_n \Vert_\h = 1$ für alle $n \geq 1$ und $\Vert T\psi_n \Vert_{\mathcal{Y}} \leq \frac{1}{n}$ für alle $n \geq 1$. Da $(T\psi_n)_\on$ konvergent ist, gibt es nach obigem Argument nach Übergang zu einer Teilfolge einen Grenzwert zu ebd. Teilfolge $\psi \in \ker(T)^\perp$ mit $\Vert \psi \Vert_\h = 1$. Dies ist ein Widerspruch, da $T \psi = \lim_{k\to\infty} T \psi_{n_k} = 0$.

    Zuletzt sei $(\theta_n)_{n \geq 1}$ eine Folge in $\im(T)$, die zu einem $\theta \in \mathcal{Y}$ konvergiert. Dann gibt es ein $\phi_n \in \ker(T)^\perp$ mit $T\phi_n = \theta_n$. Nach dem vorherigen Argument erhalten wir $\Vert\phi_n-\phi_m\Vert_\h \leq c_1 \Vert \theta_n - \theta_m \Vert_\h$, und somit ist $(\phi_n)_{n\geq 1}$ eine Cauchy-Folge und kovergiert zu einem $\phi$. Folglich ist $T\phi = \theta$ und $\theta \in \im(T)$. Also ist $\im(T)$ abgeschlossen.
\end{proof}

\section{Beispiel eines Toeplitzoperators}
Wir betrachten den Hardy-Raum \( H^2(\mathbb{T}) \), welcher ein Hilbertraum ist, auf dem Einheitskreis \( \mathbb{T} \subset \mathbb{C} \), der aus den Funktionen \( g \in L^2(\mathbb{T}) \) besteht, deren negative Fourier-Koeffizienten verschwinden. Genauer gesagt ist
\[
H^2(\mathbb{T}) = \left\{ g \in L^2(\mathbb{T}) \ \bigg|\ g(z) = \sum_{k=0}^\infty \hat{g}(k) z^k, \quad z \in \mathbb{T} \right\},
\]
wobei \( \hat{g}(k) \) der \( k \)-te Fourier-Koeffizient von \( g \) ist. Die orthogonale Projektion von \( L^2(\mathbb{T}) \) auf \( H^2(\mathbb{T}) \) bezeichnen wir mit \( P \). Wir definieren den Toeplitz-Operator \( T_{f_m} \) auf \( H^2(\mathbb{T}) \) als \( f_m(z) = z^m \), wobei \( m \in \mathbb{Z} \). Der Operator \( T_{f_m} \) wirkt auf \( g \in H^2(\mathbb{T}) \) durch $T_{f_m}(g) = P(f_m \cdot g) = P(z^m g)$. Das heißt, wir multiplizieren \( g \) mit \( z^m \) und projizieren das Ergebnis auf \( H^2(\mathbb{T}) \), indem wir die negativen Fourier-Koeffizienten entfernen. Da \( f_m(z) = z^m \) keine Nullstellen auf dem Einheitskreis \( \mathbb{T} \) besitzt (weil \( |z| = 1 \) und \( z^m \neq 0 \) für alle \( z \in \mathbb{T} \)), ist \( f_m \) invertierbar auf \( \mathbb{T} \). Somit gehört \( f_m \) zu \( C(\mathbb{T}) \setminus \{0\} \), den stetigen Funktionen ohne Nullstellen auf \( \mathbb{T} \). \( T_{f_m} \) ist also ein Fredholm-Operator.

\subsection{Kern von \( T_{f_m} \)}
Sei \( g \in \ker(T_{f_m}) \), dann gilt $T_{f_m}(g) = P(z^m g) = 0$. Das bedeutet, dass \( z^m g \in H^2_-(\mathbb{T}) \), wobei \( H^2_-(\mathbb{T}) \) der orthogonale Komplement von \( H^2(\mathbb{T}) \) in \( L^2(\mathbb{T}) \) ist und aus Funktionen mit ausschließlich negativen Fourier-Koeffizienten besteht:
\[
H^2_-(\mathbb{T}) = \left\{ h \in L^2(\mathbb{T}) \ \bigg|\ h(z) = \sum_{k=-\infty}^{-1} \hat{h}(k) z^k \right\}.
\]
Da \( g \in H^2(\mathbb{T}) \) nur nichtnegative Fourier-Koeffizienten hat, führt die Multiplikation mit \( z^m \) (eine Verschiebung der Koeffizienten um \( m \)) dazu, dass \( z^m g \) nur negative Koeffizienten hat, wenn \( g = 0 \) ist. Folglich ist der Kern trivial $\ker(T_{f_m}) = \{0\}$, also \( \dim(\ker(T_{f_m})) = 0 \).

\subsection{Kokern von \( T_{f_m} \)}
Der Kokern ist der Quotient $\operatorname{Koker}(T_{f_m}) = H^2(\mathbb{T}) / \operatorname{im}(T_{f_m})$. Um seine Dimension zu bestimmen, betrachten wir den adjungierten Operator \( T_{f_m}^* \). Für Toeplitz-Operatoren gilt \( T_{f_m}^* = T_{\overline{f_m}} \), also $T_{f_m}^* = T_{\overline{z^m}} = T_{z^{-m}}$. Es gilt $(T_{f_m}) \cong \ker(T_{f_m}^*)$. Sei \( h \in \ker(T_{f_m}^*) \), dann ist $T_{z^{-m}}(h) = P(z^{-m} h) = 0$. Das bedeutet, dass \( z^{-m} h \in H^2_-(\mathbb{T}) \). Da \( h \in H^2(\mathbb{T}) \), hat \( h \) eine Fourier-Reihe mit Koeffizienten \( \hat{h}(k) \) für \( k \geq 0 \). Die Multiplikation mit \( z^{-m} \) verschiebt diese Koeffizienten um \( -m \), sodass die nichtverschwindenden Koeffizienten nur für \( k = 0, 1, \ldots, m - 1 \) auftreten. Also ist $\dim(\ker(T_{f_m}^*)) = m \quad \text{für } m > 0$. Folglich ist $\dim(\operatorname{Koker}(T_{f_m})) = m$.

\subsection{Abgeschlossenheit des Bildes}
Da \( T_{f_m} \) ein beschränkter linearer Operator zwischen Hilberträumen mit endlichdimensionalem Kern und Kokern ist, und weil \( f_m \) keine Nullstellen auf \( \mathbb{T} \) hat, ist das Bild von \( T_{f_m} \) abgeschlossen.

\subsection{Der Index von \( T_{f_m} \)}
Der Index eines Fredholm-Operators ist definiert als $\operatorname{Index}(T_{f_m}) = \dim(\ker(T_{f_m})) - \dim(\operatorname{Koker}(T_{f_m})) = 0 - m = -m$. Die Windungszahl \( \operatorname{wind}(f_m) \) von \( f_m \) ist die Anzahl der Umläufe, die \( f_m(z) \) um den Ursprung macht, wenn \( z \) einmal den Einheitskreis \( \mathbb{T} \) durchläuft. Formal definiert ist sie durch
\begin{align}
    \operatorname{wind}(f_m) = \frac{1}{2\pi i} \int_{\mathbb{T}} \frac{f_m'(z)}{f_m(z)} \, dz.
\end{align}
Für \( f_m(z) = z^m \) erhalten wir
\begin{align}
    \frac{f_m'(z)}{f_m(z)} = \frac{m z^{m-1}}{z^m} = \frac{m}{z},
\end{align}
also
\begin{align}
\operatorname{wind}(f_m) = \frac{1}{2\pi i} \int_{\mathbb{T}} \frac{m}{z} \, dz = \frac{m}{2\pi i} \int_{\mathbb{T}} \frac{1}{z} \, dz = m.
\end{align}
Der Wegintegral \(\int_{\mathbb{T}} \frac{1}{z} \, dz\) über den Einheitskreis ergibt \( 2\pi i \). Es gilt der Zusammenhang $\operatorname{Index}(T_{f_m}) = -\operatorname{wind}(f_m)$.

\singlespacing
\nocite{*}
\bibliographystyle{plain}
\bibliography{main}
\end{document}
